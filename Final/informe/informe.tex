\documentclass[a4paper,12pt]{article}
\usepackage[utf8]{inputenc}
\usepackage{titlesec}
\usepackage[scale=0.8]{geometry}
\usepackage{graphicx}
\usepackage{xcolor,sectsty}

\pagenumbering{roman}
\definecolor{myred}{RGB}{128,0,32}
\subsectionfont{\color{myred}}
\sectionfont{\color{myred}}

\author{
        Ernandorena, Iván\\
        Legajo: E-1115/1\\
        \texttt{ivan.ernandorena@gmail.com}
}
\date{\today}
\title {
    \Huge\textsc{Trabajo Final - Agenda Personal\\}
    \large\textsc{Análisis de Lenguaje de Programación}
}

\begin{document}
  \pagenumbering{gobble}
  \thispagestyle{empty}
  \maketitle
  \begin{center}
  \includegraphics[scale=0.5]{photo}
  \end{center}

  \newpage
  \section*{Introducción}
    El trabajo final consiste en una agenda personal que además de llevar contactos como una agenda común, se
    pueden organizar reuniones, recordatorios, deudas y una lista de compras o cosas para hacer.
    \subsection*{Estructura de la agenda personal}
    La agenda consiste de:
    \begin{itemize}
      \item Un dueño: Este es quien creó la agenda y con este nombre se la guardará y se la cargará respectivamente.
      \item Contactos: Una lista de contactos de la forma \texttt{nombre-teléfono-dirección}.\\
      Por ejemplo: Martín 341-1235559 Pellegrini 200. También se pueden tener contactos solo de la forma \texttt{nombre-teléfono}.
      \item Recordatorios: También es una lista que puede tener \texttt{reuniones} y \texttt{recordatorios}.
      \item Deudas: Una lista de deudas de la forma \texttt{nombre-cantidad-razón}
      \item Compras o cosas para hacer: Por último se tiene una lista simple donde podemos escribir cualquier cosa que se desee recordar.
    \end{itemize}

  \section*{Guía de usuario}
  \subsection*{Comandos}
    En una primera instancia tenemos los comandos para abrir, guardar, etc. Estos son los siguientes:
    \begin{itemize}
      \item \textbf{:? o :help} Imprime los demás comandos con sus respectivos argumentos.
      \item \textbf{:browse} Muestra el nombre de la agenda cargada actualmente, si hay alguna.
      \item \textbf{:load $\langle$nombre$\rangle$} Carga una agenda existente, utilizando el nombre con el que se la creo.
      \item \textbf{:display} Muestra toda la agenda cargada sin filtros.
      \item \textbf{:operations} Imprime la ayuda para las operaciones sobre la agenda.
      \item \textbf{:save} Guarda los cambios realizados.
      \item \textbf{:close} Cierra la agenda actualmente abierta. También guarda los cambios si se le indica.
      \item \textbf{:quit} Cierra la interfaz.
    \end{itemize}
    Todos los comandos pueden ser abreviados con la primera letra de cada uno. Por ejemplo, \texttt{:s} en lugar de \texttt{:save}.

    \subsection*{Operaciones}
    Luego, las operaciones sobre la agenda son la siguientes:
    \subsection*{Operaciones de creacion y agregado}
    \begin{itemize}
      \item \textbf{newSched $\langle$nombre$\rangle$} Crea una nueva agenda.
      
      \item \textbf{addContact $\langle$nombre$\rangle$ $\langle$teléfono$\rangle$ $\langle$dirección$\rangle$}
        Añade un nuevo contacto. Se puede añadir un contacto que tenga solo número de teléfono. Además tener en cuenta que el teléfono 
        debe ser de la forma \texttt{característica-número}.
      
      \item \textbf{addRemind $\langle$[R][M]$\rangle$ $\langle$día-mes-año$\rangle$ $\langle$hora:min$\rangle$ $\langle$descripción$\rangle$}
      Añade un nuevo recordatorio. Recordar añadir la R (reminder) o la M (meeting) al comienzo para poder diferenciarlos.
      También utilizar bien la notación para las fechas, por ejemplo 26-09-2017 o 26/09/2017; al igual que con la hora, utilizar los dos puntos.

      \item \textbf{addDebt $\langle$nombre$\rangle$ $\langle$\$dinero$\rangle$ $\langle$descripción$\rangle$}
      Añade una nueva deuda.

      \item \textbf{addGrocerie $\langle$descripción$\rangle$} Añade un nuevo elemento a la lista de compras.
    \end{itemize}
    \subsection*{Operaciones de eliminación}
    \begin{itemize}  
      \item \textbf{delContact $\langle$nombre$\rangle$} Elimina el contacto dado.
      \item \textbf{delRemind $\langle$[R][M]$\rangle$ $\langle$día-mes-año$\rangle$ $\langle$hora:min$\rangle$ $\langle$descripción$\rangle$} 
      Elimina el recordatorio dado. Como se puede tener la misma descripción en varias fechas, o la misma fecha para distintas descripción esta 
      operación necesita todo el recordatorio.

      \item \textbf{delDebt $\langle$nombre$\rangle$ $\langle$\$dinero$\rangle$ $\langle$descripción$\rangle$} Elimina la deuda dada. También, como puede haber varias deudas de la misma persona se necesita toda la deuda.

      \item \textbf{delGrocerie $\langle$descripción$\rangle$} Elimina el elemento dado.
    \end{itemize}

    \subsection*{Operaciones de actualización}
    \begin{itemize}
      \item \textbf{updPhone $\langle$nombre$\rangle$ $\langle$nuevo teléfono$\rangle$} Actualiza el número de teléfono de un contacto existente. Recordar como se ingresan los teléfonos en \textbf{addContact}

      \item \textbf{updAddress $\langle$nombre$\rangle$ $\langle$nueva dirección$\rangle$} Actualiza la dirección de un contacto existente.
    \end{itemize}

    \subsection*{Operaciones de busqueda y filtrado}
    \begin{itemize}
      \item \textbf{searchContact $\langle$nombre$\rangle$} Busca un contacto por su nombre.
      \item \textbf{allContacts} Lista todos los contactos.
      \item \textbf{allReminds} Lista todos los recordatorios.
      \item \textbf{allMeetings} Lista todas las reuniones.
      \item \textbf{allDebts} Lista todas las deudas.
      \item \textbf{allGroceries} Muestra la lista de compras.
      \item \textbf{interval $\langle$días$\rangle$} Lista todas las reuniones y recordatorios entre la fecha actual y $\langle$días$\rangle$
      \item \textbf{thisWeek} Lista todas las reuniones y recordatorios para la semana. Equivalente a realizar \texttt{interval 7}.
      \item \textbf{thisMonth} Ídem. \textbf{thisWeek} pero para un mes.
      \item \textbf{debtsTo $\langle$nombre$\rangle$} Lista todas las deudas de $\langle$nombre$\rangle$.
      \item \textbf{debtsHigher $\langle$n$\rangle$} Muestra las deudas mayores a $\langle$n$\rangle$.
    \end{itemize}

    \section*{Módulos del programa}
      \subsection*{Types.hs}
       Aquí se definieron todas las principales estructuras utilizadas, así como también se redefinieron algunos tipos con la sentencia \texttt{type} para que luego el código sea más fácil de leer.\\
       Además de los tipos y las estructuras para la agenda, se tiene la estructura del estado que lleva el programa en todo momento. El estado consta de el último archivo cargado con el comando \texttt{:load}
       y de una instancia de la agenda cargada, si es que hay una. \\ 
       Para tener la posibilidad de no tener una agenda cargada se definió el siguiente tipo de dato:
       \begin{verbatim}
       data LoadSched = Null | LS Schedule
       \end{verbatim}
       donde \texttt{Schedule} es la estructura de la agenda definida en el mismo módulo y explicada más arriba.

       \subsection*{AST.hs}
       En este módulo se encuentra el árbol de sintaxis abstracta de nuestro lenguaje sobre la agenda.
       Simplemente son los tipos que representan los comandos y las operaciones explicadas en la sección
       anterior. Cuando se lee, por ejemplo, la operación \texttt{newSched Juan} el analizador o \textit{parser}
       devuelve algo de la forma: \begin{verbatim} NewSched Juan :: ScheduleComm \end{verbatim} Ídem. con las demás operaciones y comandos, con la excepción de que los comandos están separados en el tipo \textit{InterpreterComm}.

       \subsection*{Parser.hs y ParserComm.hs}
       Los analizadores o \textit{parsers} son los encargados de tomar los comandos proporcionados por el usuario y devolverle al programa algo con lo que pueda trabajar. También se utilizan para interpretar los archivos de las agendas. En este caso tenemos dos tipos de \textit{parsers}, en \texttt{Parser.hs} están las funciones encargadas de analizar las estructuras de la agenda, por ejemplo un recordatorio de la forma \texttt{R 24/12/2017 00:00 Navidad} es analizado por \texttt{reminderP} y le devuelve al programa algo de la forma: \begin{verbatim} Remind (DateTime 2017 12 24 00 00) "Navidad" :: Reminder \end{verbatim} Por otro lado, se tienen los analizadores de las operaciones que utilizan a su vez los analizadores de estructuras de agenda, estos se encuentran en \texttt{ParserComm.hs}.

       \subsection*{PrettyPrinter.hs}
       Este módulo se puede pensar como el caso inverso a el analizador. Se útiliza para poder mostrarle al usuario de forma mas legible 
       las consultas que realiza sobre la agenda.
       La idea es que estas funciones esten en sintonia con los \textit{parsers} de las estructuras ya que los archivos son guardados 
       utilizando funciones de \texttt{PrettyPrinter}. Si en los archivos se utilizara una sintaxis distinta de la que utiliza 
       el usuario para las estructuras deberíamos tener otro analizador para los archivos. En este caso no fue necesario al utilizar en 
       \texttt{PrettyPrinter} la misma sintaxis que en los \textit{parsers}. 

       \subsection*{Commands.hs}
       Aquí es donde tu deseos se hacen realidad! Este módulo se encarga de realizar las operaciones y comandos que el usuario ingresa desde mostrar la ayuda y cargar una agenda hasta agregar un nuevo contacto a la misma.\\
       Primero un comando es analizado por la función\\
       \texttt{interpreterCommand :: String -> IO Commands}\\
       que chequea si es un comando correcto y devuelve algo de la forma \texttt{(IComm c)} o \texttt{(SComm
       s)} dependiendo si es un comando o una operación respectivamente.\\
       Luego, ese comando es analizado por la función\\
       \texttt{handleCommand :: State -> Commands -> IO (Maybe State)}\\
        que deriva su trabajo dependiendo si es un \texttt{IComm} o \texttt{SComm}. Para esto
        tenemos las funciones
        \\ \texttt{handleInterpreter :: State -> InterpreterComm -> IO (Maybe State)} y \\
        \texttt{handleSchedule ::  State -> ScheduleComm -> IO (Maybe State)} respectivamente que realizan efectivamente la acción solicitada.

        \subsection*{Operations.hs}
        Módulo donde se encuentran todas las operaciones sobre la agenda, además de funciones auxiliares para llevar a cabo las mismas. Las utiliza \texttt{Commands.hs} para hacer cambios sobre la estructura.

        \subsection*{Main.hs}
        Donde todo comienza! Aquí es donde se maneja la interfaz de usuario y se llama a \texttt{interpretCommand} y \texttt{handleCommand} cuando el usuario ingresa un comando. \\
        Una observación interesante es que en \texttt{Commands.hs} la única operación que devuelve \textit{Nothing} es \texttt{:quit} y es cuando el búcle del intérprete termina.

    \section*{Posibles expansiones}
      Algunas funciones que se podrían agregar: %para hacerlo un sistema más adaptable al usuario son:

      \begin{itemize}
        \item Interfaz de usuario. Para un uso más cómodo para alguien que no está acostumbrado al uso de una consola.
        \item Una forma de poder eliminar los recordatorios, reuniones y deudas sin tener que repetirla. Una posible implementación sería, primero enumerar los datos cuando se muestran luego de alguna operación de filtrado. Luego, que el usuario pueda seleccionar de acuerdo a la numeración algún item y que pueda eliminar la selección de forma sencilla como por ejemplo \texttt{delRemind selection} sabiendo que se eliminará la última selección.
        \item Encriptación de los archivos guardados. Cualquiera que abra en modo lectura un archivo de agenda podrá saber la programación de tu semana y quizás no sea buena idea.
        \item Tener separadas las deudas que en dos tipos, las que son hacia el dueño de la agenda y las que el dueño debe, algo similar a lo que se hizo con los recordatorios y reuniones.
      \end{itemize}


\end{document}